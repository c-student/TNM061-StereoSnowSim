\documentclass[12pt,a4paper]{article}
\pdfoutput=1

\usepackage[utf8]{inputenc}
\usepackage[T1]{fontenc}
\usepackage[swedish]{babel}
\usepackage{amsmath}
\usepackage{lmodern}
\usepackage{units}
\usepackage{icomma}
\usepackage{color}
\usepackage{graphicx}
\usepackage{multicol,caption}
\usepackage{bbm}
\usepackage{hyperref}
\usepackage{xfrac}
\usepackage{listings}
\usepackage{multirow}
\usepackage{fancyhdr}
\pagestyle{fancy}
\rhead{\today}


\begin{document}

\begin{center}

% Upper part of the page. The '~' is needed because \\
% only works if a paragraph has started.


% Title

{ \huge \bfseries Snösimulering \\[0.2cm] }
	För domen i SGCT
	\vskip 0.4cm

% Author and supervisor
\begin{minipage}{0.8\textwidth}
\centering
	Carl Englund,
	Klas Eskilson,
	Erik Larsson,
	Daniel Rönnkvist,
	Therése Komstadius
\end{minipage}

\end{center}
\section*{Inledning}

\section*{Partiklar}
För att kunna så många partiklar som vi gjort har vi använd oss utav en teknik som kallas instansiering.
Varje snöflinga är bara två stycken trianglar som skapar ett plan, en så kallad \emph{billboard}. På grafikkorten ligger det endast en kopia av denna billboard som vi specificerar att varje partikel skall använda. Instansieringen även till att varje partikel inte får ett unikt \emph{drawcall}, utan alla ritas ut i samma drawcall.
\section*{Expanderbart}
Vi försökte satsa på att göra det så enkelt som möjligt för oss att expandera simuleringen och dess möjligheter och har därför försökt arbeta med att skapa superklasser för fält som ärver virtuella metoder för beräkningen av dess påverkan på partiklarna. Detta gör att vi enkelt kan spara alla fält i en lista utan behöva arbeta allt för mycket med varje fält.
\end{document}
